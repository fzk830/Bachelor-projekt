%\documentclass{DIKU-article}[2006/05/09]
\documentclass[a4paper]{article}

%% Language and font encodings
\usepackage[english]{babel}
\usepackage[utf8x]{inputenc}
\usepackage[T1]{fontenc}
\usepackage{courier}

%% Sets page size and margins
\usepackage[a4paper,top=3cm,bottom=2cm,left=3cm,right=3cm,marginparwidth=1.75cm]{geometry}

%\titlehead{Bachelorprojekt memory forensics}
%\authorhead{Kristian Hougaard Olsen}


\author{Kristian Hougaard Olsen \inst{1}}


\begin{document}

\subsection*{Abstract}
In this report I study processes running on a Windows Operating System (OS). I show how to acquire memory from a system, and how this data can be analysed to detect malicious activity, 
by using Volatility, as an alternative to regular antivirus software. \\
%
In the first part of the report I describe how processes on Windows are structured, and what Virtual Address Descriptors are and how these can be used to track down activity that is trying to avoid detection. \\
%
For the second part I look at different testing environments that are running Windows 7, along with applications one would expect to see on a regular system. 
The purpose of this is to discover which of the subjects has already been infected and compare this to a clean system, by using the information described earlier. \\


\newpage

\tableofcontents

\newpage

\section{Introduction}

% Motivation
% Scope
% Preventing is not realistic, since attackers are always looking for unidentified exploits - expand on this

As computers become a common tool for a wide range of jobs, they also turn into primary targets for criminals to exploit. Being the victim of malware can cause plenty of damage and(/or) cost a lot of money. 
Therefore it is of great importance that both companies and individuals protect their systems from malicious people. The most common way is through a firewall and some anti-virus program. 
In most cases an attacker would wish to remain undetected for as long as possible, circumventing the programs in place to keep a system safe, by using "unknown" exploits. 
This may result in the malware being active for a long time before the victim notices that anything is wrong, in which case it might be too late.
Due to this, alternative ways to identify an infected system becomes more attractive. One such method is the analysis of a computer's memory. 
Since all activity has to exist somewhere on the system, it means that hiding becomes more difficult (yet still possible), since an investigator can use this information similar to evidence from a real life crime scene. 
But just like real life other actions might change the data (un)intentionally. So the collection of said data becomes very important to avoid it being spoiled/ruined.
% Through this report I hope to show how knowledge about processes on Windows and the Virtual Address Descriptors can be used alongside (the tool) Volatility to discover (various) malicious activity(ies).


\newpage
\section{Processes on windows}

The description of processes in the following section refer to chapter 6 from the book "The Art of Memory Forensics - detecting malware and threats in Windows, Linux and Mac memory" 
by Michael Hale Ligh, Andrew Case, Jamie Levy & Aaron Walters.
%
On Windows a process consists of multiple parts: Threads, Handle tables, private virtual memory space - organized by Virtual Address Descriptor (VAD) trees. 
All of these are connected by a structure called \texttt{\_EPROCESS}, which also contains a large amount of different entries of various types. \\
%
The members of the structure contains a lot of information, but it would take far too long to go through every single one. Thus I will only focus on the ones that will serve some purpose, when we look for suspicious activity. 
Below is a figure of the \texttt{\_EPROCESS} structure containing the relevant entries. \\\\
%
% *Picture of datastructure, containing some of the important members(?).
%
One of the most important members of the structure is the \texttt{UniqueProcessId} which assign a unique ID to each process, and allow us to easily distinguish between them. 
Every process also contains a \texttt{InheritedFromUniqueProcessId} which contains the ID of the process that spawned it (parent). \\
The information we can obtain by examining these fields can be used to help us understand why certain processes were started. In some cases we might question why a processes was spawned, 
due to an unlikely relationship between the child and parent, which may indicate malicious activity. \\\\
% example
%
Another pair of useful members are \texttt{CreateTime} and \texttt{ExitTime}. These will tell us when a process was spawned and when it finished (or if it is still running, this field will be empty). \\\\
%
% expand further?
In order for Windows to keep track of every process it maintains a doubly linked list. The way this works is that every \texttt{\_EPROCESS} structure contains a member called \texttt{ActiveProcessLinks}, 
which is another structure called \texttt{\_LIST\_ENTRY}.
This structure links each process to the previous and next process in the doubly linked list. 
This way the Operating System and other applications can traverse the list in order to see what processes are currently running. \\
% Although this is very useful it is not a vital part of the OS, which means that processes can be unlinked from the list. % expand? 
% If an intruder manages to do this, detecting suspicious activity becomes much more difficult, since the system may appear to be operating correctly.
%
%
% The \texttt{VadRoot} member of each process contains a pointer to the root of the VAD tree. % used to organize the private memory space of the process
% This entry provides us with other means to detect malicious activity, and is especially important when the offender does not exist in the doubly linked list.

\newpage
\section{VAD}
As described earlier the VAD tree is used to organize the private memory space of a process. This contains the executable of the process, the list of DLLs/shared libraries, the stacks, heaps, and
regions containing a wide variety of data such as user input and data structures relevant to the application (such as internet history logs). \\ \\
%
The VAD tree is a self balancing binary tree, where each node represents a range in process memory. The operating system is responsible for creating and maintaining the VAD tree of a
process. 
% Because of this the OS is able to add data/information that isn't important to the CPU, such as the names of memory-mapped files, the initial protection or the total amount of pages committed in the region. \\ \\
%
% CPU doesn't NECESSARILY care about the information
On Windows 7 each node can have one of three different types: \texttt{short}, regular or \texttt{long}. With each type building on top of the previous one, by adding additional members to the structure. 
So by looking at the type of nodes in the VAD tree we can derive what is stored in the corresponding memory regions. \\
% LeftChild and RightChild - obvious
% StartingVpn and EndingVpn - page numbers used to derive the first and last pages in the target process' virtual memory. Multiply the page number by the size of the page.
% PushLock - ???
As an example the \texttt{short} node does not have a "subsection" member, which is required to store
a mapped file. This also means that a \texttt{short} node is prime for injected shell code, since it doesn't
need to exist on the hard drive. Using this knowledge we can perform our search for malware more
efficiently. \\ \\
% describe nodes in more detail ?
% VAD tags
% The "Tag" member can be used for this. It has a negative offset, which points to the "PoolTag" member of the "_POOL_HEADER". This exists in memory directly before the node. More in chapter 5.
%
In order to distinguish between the different types of nodes in a VAD tree we have to look at
the \texttt{tag} field of the structure. This member points to memory directly before the node, which
contains the \texttt{\_POLL\_HEADER} structure. Here we want to access the \texttt{PoolTag} member. This value
indicates if the node we're looking at is either of the type \texttt{short} or \texttt{long}. If the tag have the
value VadS or VadF we're looking at a \texttt{short}. The tags Vad, Vadl, Vadm represents a \texttt{long} node. \\ \\
% What about "normal" node ???
% VAD flags
Each of node type also contains a set of flags (embedded/defined by the members: u, u1, u2
and so on), with the number of flags depending on the size. Looking at the flags structure we can
collect more information about a node. \\
We can get the amount of pages that was committed by looking at the \texttt{CommitCharge} member. If
this is not present it means that the memory has only been reserved, but not paired with physical
addresses. \\
\texttt{MemCommit} can tell us if the memory was committed right away or if it was just reserved. Code
injecting malware usually commits all pages up front. \\\\
%
The \texttt{Protection} member, decides what can be done with the memory of the region. This can
be things such as reading, writing, executing, copy-on-write or disable all access to the memory. \\
% expand on this ?
The main problem with this field is that it only shows the initial protection value. This means
that it can be changed later, thus contradicting the initial value, but it won't show the new value.
This is because the member is tied to the whole memory range, which may contain multiple pages.
Thus we cannot use the value of this field to filter our search for malicious activity. \\
%
The \texttt{PrivateMemory} member is a control bit that tells us whether a committed region can be
shared with or inherited by other processes. If this bit is set it means that the region cannot
contain mapped files, named shared memory or copy-on-write DLLs. \\



\subsection{Volatility}





\newpage

\section{Experiments}

\subsection{Test environment}


For my experiments I use VirtualBox to manage a few virtual machines that I can use to perform my tests. 
This means that the setup is quite different from a real scenario in certain areas. First of the forensic part is done on the host system, instead of a dedicated computer. Ideally one would prefer that these were separated, 
to avoid any risk of infecting the system used to investigate data from the victim(s), since that could allow an attacker to corrupt our results.

To acquire memory from a virtual machine we can use an existing feature from VirtualBox, by starting said machine in debug mode.
This is done with the following command: VirtualBox --dbg --startvm "VMname". % show snip of cmd ???
Once the machine is booted up, we can enter the debug command line and use the following command: .pgmphystofile "filename.dmp".
This will write the content of the machine's memory to a file which we can analyse with Volatility.

In a real case you would have to obtain the memory of a system through some software tool, either locally through a USB or by remote acquisition over the network.

Since the focus of this report is to detect suspicious (instead of preventing it) activity after a breach, the virtual machines are not running any antivirus software and the firewall has been disabled. 
Another reason for this is that the payloads exploit known weaknesses, so any antivirus program should (hopefully) be able to detect them before any damage can be done to the system.


To infect my virtual machines I use msfvenom which is a part of Metasploit. This allows me to generate payloads targeted at the virtual machine.
% Procedure further described in relevant cases.
Both cases gives attacker a backdoor, once the malicious file has been executed (exe) or injected (dll).



















\subsection{Clean system}

Now that we know how to obtain a memory dump we can start by looking at a clean system to
get an idea of what processes we should expect to see. \\
To get a list of the processes I use the following command: \\
\texttt{volatility$\_$2.6.exe -f "filename" --profile=Win7SP1x86 pslist}. \\
This walks the doubly-linked list of \texttt{\_EPROCESSES} and we can see the values of some of the
members contained within the structure. In the first column we have the first 14 characters of
the process name. In the next we have the unique ID of each process, and the column afterwards
contains the ID of the parent process (the one that spawned it). We can see the number of handles
and threads in the following two columns. And the last two columns show the \texttt{CreateTime} and
\texttt{ExitTime} (which is usually empty). \\\\
%
% Some image showing the pslist
%
Looking at the output we can see that plenty of processes are currently running. The majority
of the processes are critical to the system, and we should expect to see them in every memory
dump that we investigate during this report. Below is a short description for most of them. \\
Some can have multiple instances running and others should only have 1. All of these should be
located in the system32 directory. \\\\
%
% Idle - a container that the kernel uses to charge CPU time for idle threads (not present in the generated list) \\\\
%
System is not a real process, it does not have an executable on disk. It is the standard home
for threads that run in kernel mode. It does have a Process ID (PID) which is always 4, and it is
the owner of any sockets or handles to files that are opened by kernel modules. \\\\
%
Csrss is a client/server runtime subsystem. It assists with the creation and deletion of processes
and threads. This process maintains a private list of objects that can be used to cross-reference
with other sources. Each session receives a dedicated copy, so multiple CSRSS processes are to
be expected. Due to this, malware might attempt to blend in by exploiting using a similar name.
The real one is located in the system32 directory. \\ \\
% Described further in chapter 12.
Services manages Windows services and maintain a list of them in a private memory space. There
should only be one instance of this process running on the system, and it should also be the parent
of all svchost.exe, along with spoolsv.exe and SearchIndexer.exe. \\\\
%
There will be several svchost processes running at the same time. These provides containers
for DLLs that implement services. Just like csrss malware may also try to use names resembling
the legitimate process, in order to make detection more difficult. The real executable is located
in the system32 directory. \\\\
%
The local security authority subsystem (lsass) process' task is to enforce security policy, verify
passwords and create access tokens. The private memory space of this process contains plaintext
password hashes. This means that it is a prime target for code injection. There should only ever
be a single instance running (located in system32). On Windows 7 the PPID of this should be
"wininit.exe". \\\\
%
Winlogon is the process that presents the interactive logon prompt, initiates the screen saver,
helps load user profiles and responds to Secure Attention Sequences (SAS), such as CTRL + ALT
+ DEL. \\\\
%
% This process also monitors files and directories for changes on systems that implement Windoes File Protection (WFP). Executable located in system32 dir. RESEARCH MORE ???
%
A system will have one Windows Explorer process (explorer.exe) for each user that is logged in.
It handles user interaction with the system, like GUI-based navigation of folders, presenting the
start menu, and more. This process also have access to documents that the user opens, credentials
used to log in to FTP sites via Windows explorer, and other sensitive data. \\\\
%
Smss is the session manager, that creates sessions to isolate OS services from various users who may
log on via the console or Remote Desktop Protocol (RDP). It is the first real user-mode process
that starts during the boot sequence. \\\\
% Described further in chapter 14.
%
Now that we know what processes we expect to see, we can compare to the results of the memory
analysis. We can see that multiple csrss.exe processes are running. \\
There is one instance of services.exe running with the PID of 424. The PPID is equal to 336, which
is the wininit.exe as it is supposed to be. \\
Plenty svchost.exe are running. They all have the PPID of 424 (services.exe), which is also the
case for spoolsv.exe and SearchIndexer.exe. These are also the values that we expect. \\
We have one lsass.exe running, with a PPID of 336 (wininit.exe). This is also correct. \\
So every process that is critical to the system is behaving as intended when examining the process
list. \\\\
%
Along with the ones described above we also have a few other non-critical processes related to
Windows. Most of them don't provide any important functionality, so they are of little importance
to the analysis. Though there are a few processes, who still performs tasks that either
help critical processes or cause a better user experience such as taskhost.exe, taskeng.exe (?) and VmiPrvSE.exe. \\\\
%
% Now since this system is a virtual machine, there are also several processes that wouldn't normally appear on a real work PC.
% VM processes(?)
% VBoxService.ex
% vmicsvc.exe (multiple)
% VBoxTray.exe
%
Finally we can see some processes that the user has started, such as firefox and notepad. Looking at the first instance of firefox.exe (and notepad.exe), we can see that the parent process ID (PPID) match the process ID (PID) of explorer.exe (2156) while the remaining three browser processes, have been spawned by the first firefox process (976). 
% armsvc.exe - component of Adobe Acrobat. non-essential.


\newpage
\subsection{Detecting malicious process}
\subsubsection{Setup} % Need better name?
In this case I will look at a system where some malicious file has been executed. \\ % further description
%
\subsubsection{Investigation} % And analysis?
Looking at the content of the process list, we can see that it is very similar to the previous example,
and nothing seems to stick out immediately. However looking closer we can see that one process
(svchost - PID = 3476) does not seem to have the expected properties. The most obvious thing is that
it has been spawned by \texttt{explorer.exe} (PPID = 2136). This clearly contradict that all
\texttt{svchost.exe} processes should have the same parent process (services = 436), which was stated
above. \\
So now we know that we're dealing with some process that is pretending to blend in by disguising
itself as another process. At this point we can't say much else about it, but if we use other
functionalities of volatility we might be able to determine what it's doing. \\ \\
By using the handle plugin: \\
\texttt{volatility$\_$2.6.exe -f "filename" --profile=Win7SP1x86 handle -p 3476 -t File} \\
% image of output?
we can get information about externals connections (pipes). \\
Since we've already discovered a suspicious process we can pass the PID as a parameter to narrow
our search. So in this case we can see 4 different pipes, but the one we're interested in is 0x58
because of \texttt{\textbackslash Device\textbackslash Afd\textbackslash Endpoint}, which indicate network activity (More in chapter 11). \\\\
% use volatility... netscan to see established connection between system and attacker (???)
Once we're aware that the process is involved in network activity we can use netscan: \\
\texttt{volatility$\_$2.6.exe -f "filename" --profile=Win7SP1x86 netscan} \\
to gain further information. Here we look for rows where the PID column is equal to the suspicious
process (3476). Looking at these rows we can see the state of the connection when the memory
was acquired, along with the IP protocol, Local Address and foreign Address (attacker).
So here we can see that a connection was established between the system (192.168.1.146) and a foreign address (192.168.1.58) through port 4444.
Currently there is an issue where the PID is showing as -1, (should be 3476). \\\\
%
\subsubsection{Discussion}
After we've gone through the process of gathering memory and analysing it to look for malicious
activity, it's worth taking a moment to reflect on the quality of this test. Since the initial
step involves walking the doubly-linked list and looking for anomalies (in this example PPID not
matching, but it could also be things like the name of a process), we're already limited to a very
black/white scenario. Either the PPID is what we're expecting (Services.exe) or it is something
different (Explorer.exe in this case). This means that we will never have any false positive error,
since a legitimate svchost process will always have the correct PPID. \\
On the other hand it is possible to have a false negative error, since the malware might have managed
to hide itself within another process by taking control of it. \\
It might also be the case that the system has been infected by something that does not even exist
on the disc (such as shellcode). Thus the test described above will not be enough to determine if
a system has been infected or not, due to the fact that important data can be tampered with to
conceal what is really going on. So we must rely on multiple sources to make sure that we're safe.
One such thing which was described earlier is VAD tree(s). \\\\

\newpage
\subsection{Detecting malicious DLL}

For the second test, we're looking at the memory from a different virtual machine. Once again we
can walk the doubly-linked list by using the pslist plugin. \\
Currently we can see that rundll32.exe (PID = 3504) has been spawned by calc.exe (PID = 1968),
which makes it a bit too obvious that something suspicious is going on. This is something I'm
looking to change later, maybe by injecting the dll into one of the svchost processes, since their
tasks are already related to Dlls. \\
It might also be caused by the method used to inject a dll into another process (powersploit). \\
(Assuming the list does not show any obvious signs of malicious activity). \\
So here we clearly see that problem with the method used in the previous case. Everything seems
to be in order (even though we know that is not the case), and thus we might be inclined to
assume that the system is safe. \\ \\
%
We're dealing with a remote DLL injection, but we don't know the name of the malicious one
(so using dlllist wont get us anywhere). Instead we can try to use either ldrmodules or malfind
to detect it. \\
Ldrmodules returns a huge list, which provides the same problem as dlllist. On the other hand
malfind only returns a list of 11 processes. Out of those, the last 4 all point to rundll32.exe \\
(PID = 3504), with different addresses. The reason why I focus on these 4 is that they all have a MZ
signature, which indicates a malicious region. \\\\
(Why? is this because example in book is XPSP3?). \\
(Should we expect to see calc.exe on the list since that was the process that we injected the DLL into?). \\\\
%
Narrowing the ldrmodules to calc.exe, we can see that \texttt{\textbackslash Windows\textbackslash System32\textbackslash oleaccrc.dll} 
does not exist in the three DLL lists (false in all), but there is a record of the VAD in kernel memory. 
So this indicate that the DLL was unlinked. The DLL seems to be legitimate, but properties seems suspicious (might be false positive). \\\\
%
\subsubsection{Discussion}
False positives should be possible, but further analysis should show that the process is not a threat. Should be shown when using malfind.
False negatives should be unlikely. If the tools are used correctly it should be possible to find malicious regions.




\end{document}
